\subsection*{gpSD00} \label{fun:gpSD00}


\textbf{Syntax} \texttt{function [out1, out2, out3, out4] =
gpSD00(X, input, target,
\\ \tab targetvariance, derivinput,
derivtarget, derivvariance, test)}

\textbf{Description}
\\ This function computes the predictive mean and variance at test input for
 the LMGP model with the covariance function as sum of covSEard
and  covNoise.
\\ Usage:
\\ {[}fX, dfX{]} = gpSD00(X, input, target, derivinput,derivtarget,
derivvariance)
\\ \tab or:
\\{[}mu, S2, muderiv, S2deriv{]} = gpSD00(X, input, target, derivinput, derivtarget,
  \\ \tab     derivvariance, test)
  \\
\\Inputs:
\\ X       ..  a (column) vector (of size D+2) of hyperparameters
\\ input   ..  a n by D matrix of training inputs
\\ target  ..  a (column) vector (of size n) of targets
\\ targetvariance  ..  a (column) vector (of size n) of variances of
 targets. Unknown variances
 \\ \tab are indicated by NaN - these are then
  replaced by the noise term in the covariance
  \\ \tab function
\\ derivinput  ..  an n by D matrix of training inputs at which we have
   derivative information \\ \tab (not necessarily the same as 'input').
\\ derivtarget  ..  an n by D matrix of partial derivatives at 'derivinput',
   w.r.t. each input
\\ derivvariance  ..  an n by $D^2$ matrix, where each row is the elements of
   the covariance \\ \tab matrix associated with the appropriate derivtarget
\\ test    ..  a nn by D matrix of test inputs
\\ Outputs:
\\ fX      ..  the returned value of minus log likelihood
\\ dfX     ..  a (column) vector (of size D+2) of partial
derivatives of minus the log likelihood wrt each of the
hyperparameters
\\ mu      ..  a (column) vector (of size nn) of prediced means
\\ S2      ..  a (column) vector (of size nn) of predicted variances
\\
 where D is the dimension of the input. The form of the
covariance function is
\\
$C(x^p,x^q) = v^2 \ exp \left( -(x^p - x^q)^T P^{-1}(x^p - x^q)/2
\right) + u^2 \ \delta_{pq}$
\\
 where the first term is the squared negative exponential and the
second term
 with the kronecker delta is the noise contribution. The P matrix
is diagonal
 with ``Automatic Relevance Determination" (ARD) or ``input length
scale"
 parameters $w_1^2,...,w_D^2$; The hyperparameter v is the
``signal std dev'' and
 u is the ``noise std dev''. All hyperparameters are collected in
the vector X
 as follows: $X = [ \log(w_1) \log(w_2) \dots \log(w_D) \log(v)
 log(u)]^T$
\\
\\ Note: the reason why the log of the parameters are used in X is that this
 often leads to a better conditioned (and unconstrained)
optimization problem
 than using the raw hyperparameters themselves.
\\
\\ This function can conveniently be used with the "minimize"  \ function to train
 a Gaussian process:
\\
\\ (C) Copyright 1999 - 2003, Carl Edward Rasmussen (2003-07-24).
 Derivative adaptation, Roderick Murray-Smith (2003-07-25). Can
now cope  with a number of derivative observations independent of
the number of
 function observations. It has separate noise level for the
derivative  observations.

\textbf{Examples}
\\ demo\_example\_lmgp\_simulation.m

\textbf{See Also}
\\ SIMULLMGP00NAIVE, SIMULLMGP00MCMC, MINIMIZE, GPSD00RAN
