\subsection*{simul02naive\_hystOut}  \label{fun:simul02naive_hystOut}



\textbf{Syntax}
\\ \texttt{function [y, s2, hyst] = simul02naive\_hystOut(logtheta, covfunc, input,
   target,
   \\ \tab  xt,lag, hystProp)}

\textbf{Description}
\\ Template for function for "naive" (i.e. without propagation of variance)
 simulation of the GP model with hysteresis on the output. Uses
routine
 gpr and works similar to SIMUL00NAIVE.
 \\
 See K. Azman, J. Kocijan, Identifikacija dinami�nega sistema s
histerezo
 z modelom na osnovi Gaussovih procesov. In: B. Zajc, A. Trost
(eds.)
 Zbornik �tirinajste mednarodne Elektrotehni�ke in ra�unalni�ke
konference
 ERK 2005, 26.-28.09., Portoro�, Slovenija, 2005, Vol. A, pp.
253--256.
 (in Slovene)
\\
\\ Inputs:
\\ loghteta .. optimized hyperparameters
\\ covfunc .. specified covariance function, see help covFun for more info
\\ input .. input part of the training data,  NxD matrix, D-th regressor
  represents the state
  \\ \tab of hysteresis
\\ target .. output part of the training data (ie. target), Nx1 vector
\\ xt .. input matrix for simulation, kxD vector, see
   construct\_simul\_input.m for more info
\\ lag .. the order of the model (number of used lagged outputs)
\\ hystProp .. hysteresis properties:
\\  \tab .limits .. [xmin xmax] values, where the state of hysteris changes
\\  \tab .DYH .. change of the output due to hysteresis
\\  \tab .state .. initial state of hysteresis
\\ Outputs:
\\ y .. mean predicted output
\\ s2 .. corresponding variances with added white noise v0
\\ hyst .. corresponding states of the hysteresis

\textbf{See Also}
\\ GPR\_SIMUL, SIMUL02NAIVE
